\chapter{Conclusions}

This thesis has presented a comprehensive study on the creation, annotation, and classification of phishing emails based on their emotional content. The development of a novel dataset, combined with a robust annotation framework, has laid the groundwork for future research in the following months.

The dataset creation process leveraged advanced LLM techniques to generate a diverse and representative corpus of phishing emails, ensuring coverage across multiple emotional categories. The annotation framework was designed to optimize inter-annotator agreement and minimize bias, resulting in high-quality labeled data suitable for training machine learning models.

Future work will focus on training and evaluating various machine learning algorithms on the annotated dataset, exploring the effectiveness of different feature extraction methods, and investigating the impact of emotional content on phishing email detection rates. Additionally, further refinement of the annotation guidelines and expansion of the dataset to include more diverse samples will be considered to enhance the robustness of the classification models and improve classification accuracy.