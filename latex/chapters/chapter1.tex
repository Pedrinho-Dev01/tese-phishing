\chapter{Introduction}%
\label{chapter:introduction}

The presence of the internet in our daily lives has become increasingly significant, transforming the way we communicate, access information, and conduct business. Over the past few decades, digital communication has evolved from a luxury to a necessity, fundamentally reshaping societal interactions and organizational operations. Among the various communication channels available today, email remains one of the most ubiquitous and reliable methods of exchanging information across the globe.

Since its inception in the early 1970s, email has evolved into an essential tool for both personal and professional communication, allowing users to send and receive messages quickly and efficiently across vast distances at minimal cost. The convenience, accessibility, and widespread adoption of email have made it an indispensable component of modern digital infrastructure. Today, email is used for everything from casual personal correspondence to critical business transactions, legal communications, and authentication processes for online services.

As such, it is now massively used across the world, with billions of users relying on email for various purposes. Recent statistics indicate that over 4 billion people use email globally, with more than 300 billion emails sent and received daily. This massive scale and the critical nature of information transmitted through email make it a prime target for malicious actors who seek to exploit vulnerabilities in email systems for their own gain. Email security has consequently emerged as a critical concern for individuals, organizations, and governments alike, as the consequences of a successful email attack can be severe and far-reaching, ranging from data breaches and identity theft to financial losses, reputational damage, and even national security threats.

The landscape of email-based threats is diverse and constantly evolving. Among these threats, phishing stands out as one of the most common, effective, and persistent methods used by cybercriminals to deceive users into divulging sensitive information or downloading malware. Phishing attacks typically involve sending fraudulent emails that appear to be from a legitimate source, such as a bank, government agency, popular social media platform, or trusted colleague, in order to lure recipients into clicking on a malicious link, opening an infected attachment, or providing confidential information such as passwords, credit card numbers, or personal identification details.

The sophistication of phishing attacks has increased dramatically over the years. Early phishing attempts were often easily identifiable due to poor grammar, obvious formatting errors, and generic greetings. However, modern phishing campaigns employ advanced social engineering techniques, psychological manipulation, and increasingly sophisticated technical methods to evade detection. Attackers now leverage artificial intelligence to craft convincing messages, personalize content based on publicly available information about their targets, and create convincing replicas of legitimate websites. The rise of spear phishing, targeted attacks aimed at specific individuals or organizations, and whaling, attacks targeting high-profile executives, demonstrates the evolution and specialization of these threats.

To combat the growing threat of phishing attacks, various email security measures have been developed and deployed over the years. These include traditional spam filters that use rule-based systems and blacklists, antivirus software that scans attachments for known malware signatures, authentication protocols such as SPF, DKIM, and DMARC that verify sender legitimacy, and multi-factor authentication systems that add additional layers of security. However, these measures are not foolproof, and cybercriminals are constantly finding new ways to bypass them. The arms race between security professionals and attackers continues to intensify, with each side developing increasingly sophisticated techniques.

One promising but underexplored avenue in the fight against phishing involves analyzing the emotional content and psychological manipulation tactics employed in malicious emails. Phishing attacks often exploit human emotions such as fear, urgency, curiosity, or greed to prompt hasty actions without careful consideration. 
This work aims to explore the defense angle against phishing attacks through the development of a machine learning-based emotion detection system that can identify and flag potentially malicious emails based on their emotional content and psychological manipulation tactics. By analyzing the emotional tone, sentiment, and affective characteristics of emails, we can gain deeper insights into the intent behind them and identify patterns that may indicate a phishing attempt. This approach represents a novel contribution to the field of email security, bridging the domains of natural language processing, affective computing, and cybersecurity.