\chapter{Introdução}%
\label{chapter:introduction}

The presence of the internet in our daily lives has become increasingly significant, transforming the way we communicate, access information, and conduct business. One of the most used ways to communicate over the internet is through email. Email has become an essential tool for both personal and professional communication, allowing users to send and receive messages quickly and efficiently.

As such, its now massively used accross the world, with billions of users relying on email for various purposes. This makes it a prime target for malicious actors who seek to exploit vulnerabilities in email systems for their own gain. Email security is a critical concern for individuals and organizations alike, as the consequences of a successful email attack can be severe, ranging from data breaches to financial losses.

Of these attacks, phishing is one of the most common and effective methods used by cybercriminals to trick users into divulging sensitive information or downloading malware. Phishing attacks typically involve sending fraudulent emails that appear to be from a legitimate source, such as a bank or social media platform, in order to lure recipients into clicking on a link or opening an attachment.

To combat the threat of phishing attacks, various email security measures have been developed, including spam filters, antivirus software, and two-factor authentication. However, these measures are not foolproof, and cybercriminals are constantly finding new ways to bypass them.

This work aims to explore the defense angle against phishing attacks based on the development of a machine learning-based emotion detection system that can identify and flag potentially malicious emails based on their emotional content. By analyzing the emotional tone of emails, we can gain insights into the intent behind them and identify patterns that may indicate a phishing attempt.