\chapter{Work Proposal}

This chapter outlines the comprehensive work plan for the 2nd semester. 

\section{Work Plan and Timeline}

The project is structured into four main phases, each designed to build upon the previous stage and ensure a systematic approach to development. Each phase is allocated one month for completion, allowing for thorough implementation, testing, and refinement.

\subsection{Phase 1: Phishing Detection Model Development}

The initial phase focuses on developing a robust classification model capable of detecting phishing emails based on linguistic characteristics extracted from the text content. Leveraging the existing dataset mentioned in the previous chapter, this phase includes:

\begin{itemize}
    \item Dataset Analysis and Preprocessing: Conducting exploratory data analysis on the existing dataset to understand its characteristics, distribution, and any preprocessing requirements for optimal model performance.
    
    \item Feature Engineering: Extracting relevant linguistic features including word frequency patterns, grammatical structures, URL characteristics and urgency indicators commonly found in phishing attempts.
    
    \item Model Selection and Training: Evaluating multiple machine learning algorithms and models to identify the most effective approach for phishing detection.
    
    \item Model Evaluation: Conducting rigorous testing using appropriate metrics (accuracy, precision, recall, F1-score) and cross-validation techniques to ensure model reliability and generalization.
    
    \item Parameter Optimization: Fine-tuning model parameters to achieve optimal performance while avoiding overfitting.
\end{itemize}

\subsection{Phase 2: Sentiment Analysis Model Development}

The second phase concentrates on building a specialized sentiment analysis model designed to identify emotional patterns characteristic of phishing attacks. Working with the existing dataset, this phase encompasses:

The second phase concentrates on expanding the capabilities of the previously developed model by integrating sentiment analysis to identify emotional patterns characteristic of phishing attacks.

\begin{itemize}
    \item Sentiment Annotation Review: Review and validate the sentiment annotations in the existing dataset to ensure accuracy and consistency.
    
    \item Emotional Pattern Identification: Analyzing phishing emails to identify common emotional triggers that attackers exploit.
    
    \item Model Architecture Design: Expand the architecture of the phishing detection model to incorporate sentiment analysis capabilities.
    
    \item Training and Validation: Training the sentiment model on the curated dataset and validating its ability to detect manipulative emotional patterns.
\end{itemize}

\subsection{Phase 3: Web Application Development}

The third phase involves creating a user-friendly web application that serves as the interface for the phishing detection system.

\begin{itemize}
    \item Frontend Development: Designing and implementing an intuitive user interface that allows users to upload .eml files or paste email content directly into the application.
    
    \item Backend Implementation: Developing the server-side logic to process email content, invoke the classification models, and return results to the user interface.
    
    \item Results Visualization: Create clear and informative displays of classification results, including phishing probability scores, sentiment analysis outcomes, and explanations of detected suspicious patterns.
    
    \item Security Implementation: Ensuring the web application follows security best practices, like input validation and secure data handling, to protect user data.
    
    \item User Experience Testing: Conducting usability tests to ensure the application is accessible and easy to use for individuals with varying technical backgrounds.
\end{itemize}

\subsection{Phase 4: Review and Quality Assurance}

The final phase is dedicated to comprehensive review and refinement of all project components.

\begin{itemize}
    \item System Integration Testing: Performing end-to-end testing of the complete system to identify and resolve any integration issues or performance bottlenecks.
    
    \item Model Performance Review: Conducting final evaluations of both models using additional test datasets to verify their effectiveness and reliability in real-world scenarios.
    
    \item Documentation Completion: Finalizing all technical documentation, including system architecture descriptions, model specifications, API documentation, and user guides.
    
    \item Code Quality Assurance: Reviewing all source code for adherence to best practices, proper documentation, and maintainability.
    
    \item Thesis Document Revision: Thoroughly reviewing and editing the written thesis document to ensure coherence, clarity, and academic rigor throughout all chapters.
    
    \item Final Presentation Preparation: Preparing materials for the final project presentation, including demonstrations of the web application and discussion of findings and contributions.
\end{itemize}

\section{Project Timeline}

Figure~\ref{fig:gantt} presents a Gantt chart illustrating the timeline for all project phases. Each phase is allocated one month, ensuring adequate time for implementation, testing, and documentation while maintaining steady progress toward project completion.

\begin{figure}[htbp]
    \centering
    \begin{ganttchart}[
        hgrid,
        vgrid,
        x unit=2.5cm,
        y unit title=0.6cm,
        y unit chart=0.8cm,
        title height=1,
        bar/.style={fill=blue!50},
        bar height=0.6,
        group right shift=0,
        group top shift=0.6,
        group height=.3,
        group peaks height=.2
    ]{1}{4}
        \gantttitle{Project Timeline (February - May 2026)}{4} \\
        \gantttitle{Feb}{1}
        \gantttitle{Mar}{1}
        \gantttitle{Apr}{1}
        \gantttitle{May}{1} \\
        
        \ganttbar{Phase 1}{1}{1} \\
        \ganttbar{Phase 2}{2}{2} \\
        \ganttbar{Phase 3}{3}{3} \\
        \ganttbar{Phase 4}{4}{4}
    \end{ganttchart}
    \caption{Gantt chart depicting the four-month project timeline with sequential phases}
    \label{fig:gantt}
\end{figure}

\section{Expected Deliverables}

Upon completion of the project, the following deliverables are anticipated:

\begin{enumerate}
    \item A trained phishing and sentiment analysis model with documented performance metrics
    \item A functional web application with source code and deployment documentation
    \item A complete thesis document detailing the research, methodology, implementation, and findings
\end{enumerate}